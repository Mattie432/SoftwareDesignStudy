\documentclass{article}
\usepackage{../myPackage}


\begin{document}
\pagebreak
    \section{System Overview}
    \subsection{Emergency Services App}
    The Emergency Services App is aimed towards all members of the public with access to a smartphone. The app is a companion for the traditional 999 call procedure and adds a multitude of features and extensions which can benefit both the user and the call receiver.
    \subsection{CPR App}
    The CPR App is aimed towards people who are already trained in CPR and first aid. The app is intended to be used passively, installed once and forgotten about, until it detects that there is an emergency situation near by that the individual could help at. The idea is that an individual trained in first aid can be called upon to act as a first responder until the emergency services arrive.
    \subsection{Why we chose two separate Apps}
    Initially we were designing our system as one app containing both of the above feature sets, however we soon realised in the design stages that this was impractical. Parts of one feature were starting to encroach on areas of the other and that was when we realised that the two use cases lend themselves better to separate apps. It is also true that users may only want one of the two use feature sets, for example if the apps were combined everyone would be carrying around the ability to be contacted  via the CPR system. This is impractical and a waste of resources.\\
    
    We believe that separating the features into separate apps will not inconvenience the majority of users. Those who wish to participate in the first responders initiative need only download a small app which they can setup once and 'forget about' until they are contacted with an emergency near by.
    \subsection{Servers}
    Our apps connect to our own remote servers which are used as the main 'hub' for all communication between the client devices and the emergency dispatchers. In some cases connections to these servers are direct and in others we are using the Google Cloud Messaging service to efficiently send data back and fourth.
    \subsection{API}
    We have developed an API along side our apps so that external companies can easily integrate our products into their existing systems. We decided to go along this route as we do not know enough about the potential systems our system could be installed in to be able to integrate into them ourselves. An API also allows our system to be implemented in other systems that we might not have considered during the planning stages of our project making it more accessible.
\end{document}