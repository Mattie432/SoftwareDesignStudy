\documentclass{article}
\usepackage{graphicx}
\usepackage[top=2cm,bottom=2cm,left=3cm,right=3cm]{geometry}
\usepackage{multicol}



\begin{document}

    \section{Introduction}
	According to the Ronald Arkin, the author of ‘Behaviour-Based Robotics’, an Intelligent Robot is ‘a machine able to extract information from its environment and use knowledge about its world to move safely in a meaningful and purposive manner’ [1]. Mobile robots (robots that are capable of locomotion), are one such example of an ‘intelligent’ robot. They are regularly used in many applications: from aiding disaster recovery efforts in mines and after earthquakes, to military uses such as roadside bomb detection. Recently, mobile robots have been developed for consumer applications, such as the Roomba, and wheeled mobile robots have even been developed to explore the surface of Mars and are poised to return to the moon [2].

	The ecological approach to Robot design is one in which the Robot’s environment and goals are the guiding principle of the design process [1]. This paper presents an ecological approach to the design and analysis of a mobile robot to solve the problems outlined by the AAAI Mobile Robotics Competition,1996: “Call a Meeting”.

	\subsection{Background and Prerequisites}
	This paper assumes very basic familiarity with the principles of robotics; key elements will be elucidated in layman’s terms as appropriate. To the extent that it is needed to describe necessary algorithms, mathematical notation (to the level of an undergraduate engineering or science course) will also be presented.

	\subsection{Outline of the paper}
	Section 2 presents a review of relevant literature as it pertains to robot control architectures, localisation methods, motion planning, and feature recognition; starting from the broad and converging to those areas specific to the problem domain. The design of the solution our team adopted is then detailed and analysed in Section 3 and experimentation and results are presented in Section 4. Section 5 concludes with a brief concluding discussion of the task, directions for future teams, and acknowledgements to those individuals whose guidance and advice was invaluable during the design process.
\end{document}