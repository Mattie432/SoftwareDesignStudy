\documentclass[]{article}
\usepackage{../myPackage}



\begin{document}
\pagebreak
\section{Modules}
    \subsection{Emergency App}
    	\subsubsection{Video Streaming Rationale}
    	Video calling is not a recent development, but one that has been processed throughout time [33]. It has finally emerged and is making quite the splash in the technology world. Mobile phones with video calling capabilities provide instant face-to-face communication with anyone, anywhere. In order to have this capability, the phone must have a camera and a stable internet connection.\\
    	
    	Nowadays, most mobile phones have built-in cameras [34][35] with the capability of capturing images with a resolution of at least 2 MP, which is of HD quality or transferred into pixels: 1600x1200. Some mobile phones provide even more than 41 MP resolution of a single image. These cameras, however, record video with a lower quality than single images. Older phones with older 2 MP camera sensors can record video with resolution of 240p, which is equal to 320x240, at 15 frames per second. With the advances of technology, this has changed a lot. Nowadays, most mobile phones can record videos with at least 480p (640x480) resolution and at least 30 frames per second. The newest models from 2014 even have the possibility of recording video with 4K resolution (4096x2160) at 30 frames per second. Apart from high quality of the video, modern mobile phones can exchange quality for speed and capture video with 720p (1280x720) at 120 fps or some even reach 240 fps.\\
    	
    	Another feature that is required in order to make a video call and all mobile phones with built-in cameras have is mobile or wireless internet connection. Mobile connection is usually slower, starting with 2G, 3G and reaching the newest and the most advanced 4G. As of June 2014, the 2G and 3G coverage of the UK premises is 99.7\% and 99.5\%, respectively [36]. Nevertheless, further technology improvements are being developed. 4G coverage has already reached 73.0\% of UK premises and researchers has began work on 5G, which is believed to be introduced in the early 2020s [37].\\

\begin{table}[h]
\centering
\begin{tabular}{|l|l|l|}
\hline
\textbf{Technology} & \textbf{Coverage} & \textbf{Speed}   \\ \hline
2G (GPRS)           & 99.7\%            & 56 to 115 kbps   \\ \hline
2G (EDGE)           & 99\%              & up to 237 kbps   \\ \hline
3G                  & 97\%              & 0.2 to 28 Mbit/s \\ \hline
4G                  & 73\%              & 100 Mbps         \\ \hline
\end{tabular}
\end{table}

    	Based on statistics by Skype [38], a normal internet call requires a minimum of 30 kbps of download and upload bandwidth. In order to be able to conduct a simple video call using a smartphone, a bandwidth of minimum 128 kbps upload and download speed is needed, whereas 300 kbps is recommended. For higher quality of the call the speed requirements rise to 400kbps and 1.2 Mbps for the highest quality. Youtube suggests similar network requirements for its streaming service [39]. Taking into account the use of newer encoding techniques, the network requirements can be compared with the network speed that mobile technology provides. Using 2G, a simple video call can be made, with low quality settings. However, 3G speeds are enough in order to perform a medium to high quality stream and 4G will be able to provide even better streams.\\
    	
\begin{figure}[H]
		\centering
		\includegraphics[width=1\textwidth]{"SDS Video Bandwidth"}
		\caption{The bandwidth requirements for video streaming.}
	\end{figure}
	
	Wireless internet connection can be compared to the same network requirements. As it can usually provide better and more reliable internet connection, it would be the prefered way of transferring the video stream data. However, free wireless internet connection is available in limited places, thus, a combination between wireless and mobile internet connect is required, in order to provide internet access of the highest quality and coverage.\\

Thanks to the low bandwidth requirements and fast mobile network that is available almost everywhere, it is now possible to make video calls using your mobile phone wherever you are. Furthermore, with the technology evolving so fast, it is likely to be able to make High Quality video calls from everywhere in the near future.\\

These technology innovations could change how emergency situations are handled. Being able to send real time video footage to an emergency operator or an emergency team, could help them assess a given situation better and give better instructions to the people requesting help. In a phone conversation we held with a personnel of an emergency department, we asked to express what are her thoughts on introducing video streaming feature to an emergency call. ``That would be very useful. One of the issues we always have, when we take a 999 call, is that we are blind. We are going on what our caller is telling us, so to be able to see it, would make things a lot easier and we could assess the severity a lot quicker.'' - Dawn Whelan. This was also confirmed by a survey that we have conducted, which showed that 82\% of the people, who participated, think that they could benefit from such a feature.\\

	    \subsubsection{Chat System Rationale}
	    As of 2012, there are more than 32,000 registered phones with the Emergency SMS service [52]. However, according to statistics [53] there are more than 10 million people with some kind of hearing loss and 800,000 of them are severely or profoundly deaf. That means that only about 4\% of the people, who might experience the need to use the Emergency SMS service, are registered. Including a live chat feature in the emergency application will greatly reduce the difficulty of using such a service and will make it more accessible to a greater part of the population.\\

Using a live chat requires a minimum amount of internet access, which means that the chat will be available for use even in regions where mobile network coverage is poor. Moreover, such chat will be able to deliver messages in the same order they were sent, keep track whether a message has been delivered and read by the operator and will make requesting more information easier and faster. A single message will take only milliseconds to be delivered, rather than the slow speed of delivering SMS messages. If a message fails to be delivered, the application can keep trying or notify the user that it was unable to send the message. Giving such information to the users, rather than suggesting to them to resend the message if no response has been received in 3 minutes, gives them the opportunity to react faster.\\

	\subsubsection{Automatic Video Sending Rationale}
	A problem that may arise during a 999 call is that the caller may not be able to describe in details where he is located. This could be due to him being in an unknown place or just confused because of stress. The results of such a problem can be that the emergency team head in the wrong direction or arrive to the place of the emergency, but are unable to find the exact location of the accident. In order to help with transferring the exact location of the person, requesting emergency, mobile phones can be used.\\

As already stated, nowadays, all mobile phones have the ability to access internet, through various technologies. Apart from that, most mobile phones have GPS sensors [34][35] and are able to locate themselves within seconds. The acquired location can either be exact, within a few meters, if a GPS sensor is available and a good GPS signal is present, or at least identifying the correct region, based on mobile network reception. This information is also very small, in terms of data size, which means that it can easily and quickly be sent over the network. Including such a feature in the emergency application can greatly reduce the time needed for the emergency team to head in the right direction, thus, the team will be able to arrive faster.\\
    	
    \pagebreak	
    \subsection{CPR System}
    	\subsubsection{Rationale}
    	Cardiovascular diseases are one of the leading causes of death in the western world [1]. They come with an increased risk of cardiac arrest, of which there are an estimated 60,000 [2] incidents out of hospital annually in the UK - about 1 every 9 minutes. In fact, across the whole of Europe, 1 person per 1000 population will suffer a cardiac arrest in any year.\\
    	
The truth however, is that in most cases of out of hospital cardiac arrest the chances of survival are depressingly low. The average overall survival rate for England is just 8.6\% [2] and in some parts of the country, just 1 in 14 people survive an unanticipated cardiac arrest [3]. This is poor by international standards, with some of the highest survival rates being Norway (25\%), Holland (21\%) and Seattle (20\%) [2] which shows clear potential for improvement in the UK.\\

The most effective way to increase a person’s chances of survival from cardiac arrest is to perform immediate CPR, whether that be only chest compressions or mouth to mouth. Evidence suggests that where CPR is attempted, survival rates are doubled [5] and this could be expected to save around 300 lives per year. This is because the chance of survival after a cardiac arrest reduces by around 10\% every minute without proper care [6] due to the lack of oxygen the body (and especially the brain) experiences. Early CPR until paramedics arrive is very important to maintain blood circulation to the heart and brain, which also increases the chance that treatment with defibrillators is successful [6].\\

One of the main reasons that the fatality rates of the UK are so high is because of a low rate of initial CPR by bystanders: ``Fewer than one in five people who suffer a survivable cardiac arrest receive the life-saving intervention they need from people nearby'' [3]. Compare this to 73\% in Norway [4] and it is clear that this is an area for major improvement. There are several factors thought to be responsible for low levels of bystander-initiated CPR, including lack of training and fear of litigation [5]. In addition, the number of population trained in CPR is currently 3.8m [5] (out of 60m). This small proportion means the likelihood of someone being nearby when an individual suffers a cardiac arrest is very low, and furthermore, training new people to give CPR will only help in the long term as it will take some time before any difference to this likelihood is noticed. In the meantime, a better way is needed to link those trained to give CPR to those in need of it.\\

So few recoverable cardiac arrests are survived mainly due to the time it takes between the arrest occurring and the rescue attempt beginning. With the average waiting time currently around 8 minutes [7], and recent news of longer response times from emergency services [24], we need to look into ways of helping those affected before emergency services arrive.\\
		\subsubsection{Project SMS-livräddare}
		
		There is an ongoing research project in Sweden called ``Project SMS-livräddare'' [6], which aims to improve cardiac arrest survival rates by getting trained civilians to start CPR early, before the ambulance arrives [6]. Currently active in the entire capital city of Stockholm, the trial started in May 2010 and has seen massive uptake from the public, with 9,600 residents currently registered [9]. This has resulted in SMS-livräddare-volunteers reaching victims before ambulances in 54\% of cases and has helped increase survival rates from 3\% to nearly 11\%, over the last decade [8].\\

The system works by having willing civilians trained in CPR register to help if a cardiac arrest happens in their vicinity. When an emergency call is received, the geographical position of the caller is determined. If there is suspicion that a cardiac arrest has occurred the emergency operator activates a positioning system that locates the mobile phones of helpers connected to the service. In cases where a lifesaver is nearby, they are alerted via their mobile phone. Meanwhile ambulance and emergency services are alerted. The alarm to the SMS-lifesaver’s mobile phone comes as an SMS. The text message contains information from the emergency services about where the suspected cardiac arrest has occurred and the message also includes a map link which can be used to more easily find the location. The SMS-lifesaver also receives an automatically generated phone call to alert the user that an SMS arrived on the phone [6].\\

This is the basis for our implementation of this feature. We believe that this project has done incredible work and shown people a new way of being a part of first aid assistance, but that there are areas which could be refined and improved. For example, the project currently has users register two areas that they will likely be, one for day and one for evening, then uses these areas to determine people to contact. This, and other problems like it, will be what we intend to address in our implementation of the system.\\

\pagebreak
    \subsection{API}
    As it was previously explained the project consist of several systems communicating and interacting with one another. The communication between different systems is done via an Application Programming Interface (API). Since the current systems declined to share their API, we have decided to create and provide an API to third party developers. They should be able to integrate it seamlessly in their products.\\

It was decided that all interactions with the system, where possible, will be made through the API. This includes the communication between the backend user interface and the Server infrastructure as well as the communication with the mobile applications, where possible. Using a well documented API throughout the project ensures that all functional capabilities are exposed. This in turn means that if the existing emergency systems decide to integrate with our product the integration could be done without a modification of the existing infrastructure.\\

When researching different API technologies a number of requirements had to be taken into account.
\begin{enumerate}
\item \textit{The API needs to be platform independent}\\
Since the the system being developed has Android application, Server infrastructure, Operator front end, one of main requirements of the API is to be cross-platform.

\item \textit{The technology should be popular}\\
Given the variety of the systems involved, the technology used to develop the API should be well established so that properly developed and tested libraries can be used.
Since popular technologies are well known to developers, this manner of implementation will ease third-party integration with the platform.
\end{enumerate}
\end{document}
