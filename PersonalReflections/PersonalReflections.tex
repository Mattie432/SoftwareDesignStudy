\documentclass[]{article}
\usepackage{../myPackage}

\begin{document}
\pagebreak
    \section{Personal Reflections}
    \subsection{George Brighton}

    \subsubsection*{Teamwork}

    After suggesting the idea for our project, I implicitly became the team leader. During the research phase, very little management was necessary, with good progress being made by all team members contacting various would-be stakeholders, and building up a detailed picture of the current state of affairs. However, the realisation of CommandPoint came as an unpleasant surprise. The effort went from building a new system to amending one being phased in, which we quickly realised was infeasible due to Lockheed Martin's unwillingness to intimate the system's finer details. As a result, we transitioned to building two applications to co-exist with CommandPoint, with minimal integration. Matt and I worked on a CPR system, with Martin and Robbie creating an emergency mobile app to aid 999 calls, and Deyan constructing an API.

    Management became hard, especially during the second semester as other deadlines and projects piled up. Fortunately, effective use of Trello and Google Docs kept everyone's work in sync, while a Facebook group I created was invaluable for quick questions to avoid log jams. Work may have been slow on occasion, but we always knew where we were falling behind. Certainly, the weekly meetings and occasional Skype conversations helped immensely; Software Design Study has nailed home to me the importance of keeping track of progress and looking ahead to see what work remains to be done.

    If I were given the opportunity to go back to the point where our project split into two separate applications, I would delegate responsibility for the emergency app to Martin or Robbie, and keep checking up on its progress, as well as asking for updates on the API from Deyan. I think it is fair to say that my overly hands-off approach led to a certain amount of complacency amongst everyone in the team, and caused unnecessary delays.

    \subsubsection*{Activities}

    I found the process of investigating how the emergency services organise themselves extremely interesting, and relished the opportunity to look in detail at a real, deployed piece of software. On one hand, I was astounded by how dysfunctional the existing system was - initially we had many fantastic ideas - but it quickly became apparent that these were as hard to translate into a production system as they were innovative, especially in the case of something as critical as the technology behind the emergency services.

    In particular, it was humbling to see how willing various people were to help us. I have extended particular gratitude to Carol Hunter at Northrop Grumman and Mike Brady at Lockheed Martin, who were instrumental in gathering information and opinions on CAD and CommandPoint. Their contributions made us aware of the inevitability of CommandPoint, and helped steer us towards our final choices of application.

    \subsubsection*{Summary}

    This module has been without a doubt the most eye-opening in terms of commercial use of software. Very distinct from any other I have taken this year, it has provided me with a much better idea of how systems comes to exist, and the challenges that must be overcome before even a single line of code is written. In the same way software engineering investigates how systems are implemented, I see Software Design Study as an equivalent for how system ideas come to exist in the first place.

    Despite several hiccups, we worked surprisingly consistently and everyone made a valuable contribution to the project. Some things were done very well, while others have been a fantastic learning experience. Overall, I would happily take the module again.

    \pagebreak
    \subsection{Matthew Flint}
    Our first challenge was picking a topic that could sustain us throughout the entire year. We came up with a number of good ideas but ultimately decided to settle on a system for the Emergency Services as we had some vague knowledge that the current one was not very good and we thought it provided enough scope to really be able to explore.\\

I initially thought that with such a large system currently in place and it being used in a very publicly central service that it would prove relatively easy to research what the current system did, how it worked and any problems associated with it. This was not the case, obtaining information of this kind proved extremely difficult and very little of it is accessible via the internet (making the task even harder). Eventually we did find that the current system was so bad that there was already another (command point) being developed to take its place.\\

This placed us in a dilemma, from which we took a long time to decide how to proceed. We were unsure whether we wanted to build a system better than the current one, though as command point was already in development this seemed rather pointless. Then we looked at improving on command point, however it seems that in the most part they designed and developed the system well and left us little scope to do this. Eventually we decided that we should improve upon the system by thinking of new and innovative features that could benefit people using the system on both ends of an emergency. This proved to be a good direction to proceed in as it gave us a large scope to think of new ideas to help the current system and allowed us to be more creative in our design.\\

We now had an idea of what to do but were not very sure about how to start the project. We wasted a long time deliberating with no real organization until we finally decided to employ an agile scrum approach and delegate certain tasks to different people. Once this was done and we had a clearer idea of the direction we were heading in things started to move along but very slowly. This project took a long while to get going but once it did finding the next task to do was easy.\

The amount of time and effort put into producing this document was considerable, this is the project that I have been a part of and easily ate up hours on weekends especially towards the end of the project.\\

Overall I feel that we have produced a good report that a software development team could take and produce a system from. The project was not without its challenges but we have overcome them as a team and made it through.

    \pagebreak
    \subsection{Deyan Genovski}
    When the module was initially introduced to me I was confused as to what type of system we had to design. After having some initial meetings with my teammates we decided that a community project would be both interesting for us and suitable for the type of documents we had to produce. We had a couple of ideas but eventually we chose to investigate and improve the way that the emergency services currently work.\\

At the beginning of the first term our task was to research and investigate how the current emergency services work. During this research process we had to identify possible errors or flaws with the current way of managing emergencies.\\ 

As it turned out, a lot of the information regarding the emergency services available on the internet was either outdated, inaccurate or extremely vague. This in terms meant that we had to get in touch with some of the people that use or develop the existing infrastructure. This was challenging for us, because as we realized some of the people we were trying to contact were extremely busy.\\

While researching the way the emergencies are managed we found that there were a lot of flaws. There is however a system called Command Point that is actively being developed at the moment that would solve many, if not all, of the issues with the current system.\\

At that point we realized that it would be a rather wasted exercise to try and fix what is already fixed by Command Point. That is when we had to think in way that the developers of the current systems haven’t.  We decided to use the technologies provided by modern mobile phones in order to help dealing with emergencies.\\

This whole process was challenging for me but taught me that I always have to try and look at a problem from a different perspective. Even though that some perspective may seem ridiculous at first, it is surprising how often it turns out to be beneficial.\\

During the second term we had to design the systems that we chose. There were a lot of challenges both when organizing ourselves and when designing the actual product. \\

While designing our product I improved my understanding of the technologies that we had to use as I had to carefully read the documentation and produce written justification as to why we have chosen a specific technology. I felt that at times it was more difficult and time consuming to write a justification for using certain technology than to actually implement it.\\

Even though that I am person that prefers developing rather than planning, this module helped me understand that designing a software system is a difficult but important process.\\

Looking back at the project, there are some things that I would do differently but I am quite happy with what we have produced.

    \pagebreak
    \subsection{Martin Mihov}
    At the beginning of the module, I had a really vague idea about what was intended to happen during the year. While choosing a topic, we all came up with good ideas and we all wanted to focus on a community project, rather than a commercial one. This helped us identify and agree on a topic quickly. After doing a bit of research, we found out that the current way that the handling of the emergency calls works is outdated and has a lot of issues.\\

Having decided on a topic, we did further research and we managed to identify even more and more issues with the emergency services systems. By the middle of the first term, I was a bit more confident that what we were doing was going to be interesting, as well as challenging. It turned out to be both, but rather more challenging than I first thought. I found out that researching a topic was not as easy as I imagined. And this was especially the case with the emergency services, as most of the information online was outdated. This caused us to contact people working at the emergency services in order to actually get some feedback on the current systems.\\

Having identified many issues, we were ready to start thinking of solutions to the problems. The various technologies that I have learned during my time in university and I faced during my work experience, helped me to identify many solutions. Some of them turned out to be not very efficient, however, others were really innovative. By the end of the first term, we managed to focus on specific issues and choose several solutions that we were going to include in our software design, with which I was really happy.\\

During the second term we started actually designing our solutions and as we were not sure what we were going to end up with, we decided to go for an iterative approach. During my time in university, I learned about several software engineering approaches, however, iterative approaches were not famous with producing quality documentation. This made me do research on what was the best way to keep our approach iterative, while keeping a good log, which I could later use in the final report.\\

Apart from that, while designing the technical details of the solutions we chose, I did a lot of research of available technologies which were required for a specific part of the solution. This improved my general knowledge about how various things work and what are some of the state-of-the-art technologies out there. Documenting and evaluating the choices that we made through the second term, improved my confidence with software design.\\

Working in a team was another challenging task that I faced during the module. Being in a team of 5 people meant that we had to organize ourselves and split the tasks, so that each of us was assigned an equal amount of work. We tried to use everyone’s best skills in order to do what was best for the project. Having regular meetings and ensuring that everyone was happy with what was going on and the decisions we made, was very important, in order to work efficiently. In some cases this was very difficult, as some of the tasks were totally unrelated and sometimes I felt like we were working on different projects. However, in my opinion, everyone from my team did very well, showing impressive knowledge and skills in some part of the module and showing weaknesses in others.\\

Looking back at the project, I see a lot of things that could have been done better, both in terms of organization and in terms of software design approach. However, identifying those mistakes, make me feel confident about the skills that I have gained throughout the module. With this experience I will be able to be more efficient in my future projects. Apart from that I will be able to identify possible problems or conflicts earlier during a project, which will give the opportunity to deal with them easier and with less impact to the project.
    \pagebreak
    \subsection{Robert Zlatarski}
    At the beginning of the module, we were given a task to choose a project, on which we should later on produce a software design. We had some good ideas, but we all agreed that we should focus on something that could improve people’s lives. Then we came across on how emergency situations are handled nowadays and saw a lot of weaknesses and the use of outdated technologies. We decided that we should create a product, which would ease up and improve how emergencies are handled. There was still the question, should we improve the current system or create one of our own. On mutual agreement, a new system was chosen to be built, which had to be easily extendable.\\

Doing a lot of research, conducting different surveys and speaking with professionals, we faced different problems and factors that could impact the functionalities of our system. I did not have any idea of how emergencies were handled in the UK and I learned a lot of new interesting stuff on the current systems. We all split the work equally and had regular meetings two or three times a week to discuss our progress and combine our ideas. Eventually, by the end of the first term, we had a clear vision on what we were going to build, having accepted a lot of different innovative ideas and refused others.\\

At the beginning of the second term, we had to start writing up the documentation. We faced problems with choosing the software design methodology, as we had to pick either an iterative agile approach or waterfall model. Eventually, having put a decent amount of research, we decided that we should use an iterative approach. Then we started working and researching the technologies that we were going to use. It was not easy to pick the right technologies for our product. Having gone through a lot of studies and tests, we had pick those, which would suit our project the best.\\

We all put a decent amount of work on the document that we produced, facing different problems and learning interesting new stuff. Sometimes, organizing ourselves was not easy, as we had other university duties and some people were busier at a certain time than others. Nevertheless, we were spending quite a lot of time working on it, especially towards the end of the term. Working in a team of five people was sometimes a bit challenging to me. Sometimes our documentation was a bit inconsistent and those were the main things, which we discussed and did in the meetings we had. Having eventually handled that, we were making a good progress. By the end of the Easter break, most of our documents were ready and had to be structured together.\\

Overall, I am happy with the product that we have created. It turned out to be a system that could really save people’s lives and improve the current way of how emergency situations are handled. I have gained a lot of useful experience and improved both my team working and software designing skills. Moreover I am aware of technologies, which before I haven’t even heard of.

\end{document}
